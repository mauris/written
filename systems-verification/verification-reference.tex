\documentclass{article}
\usepackage[utf8]{inputenc}
\usepackage{geometry}
\usepackage{amsmath}
\usepackage{dirtytalk}
\usepackage{graphicx}
\usepackage{enumerate}
\usepackage{hyperref}
\usepackage{listings}
\usepackage[nodayofweek]{datetime}
\longdate
\usepackage{amsfonts}
\usepackage{multicol}
\usepackage{amsthm}
\usepackage{comment}
\usepackage[font=footnotesize,labelfont=bf]{caption}
\usepackage{float}

\theoremstyle{plain}
\newtheorem{thm}{Theorem}[section]

\theoremstyle{definition}
\newtheorem{defn}[thm]{Definition} % definition numbers are dependent on theorem numbers
\newtheorem{exmp}[thm]{Example} % same for example numbers

\setlength\columnsep{30pt}

\geometry{
 	a4paper,
	total={170mm,257mm},
 	left=20mm,
 	top=20mm,
}

\lstset{basicstyle=\footnotesize\ttfamily,breaklines=true}

\title{
	 \huge 303: Systems Verification\\
	 \huge -- Reference --
}
\date{\today}
\author{
	Sam Yong \\
	\small \href{mailto:sam.yong17@imperial.ac.uk}{sam.yong17@imperial.ac.uk}
}


\begin{document}
\maketitle

\begin{multicols}{2}

\paragraph{} NOTICE: This set of reference is currently incomplete with respect to the syllabus of the course and will be updated as the semester progress. By end Mar 2018 it should be complete.

\section*{Foreword}  

\paragraph{} This reference was made as an condensation from the lecture slides and notes provided by Prof. Alessio R. Lomuscio, Prof. Michael Huth and Prof. Mark D. Ryan in the Imperial College London, Department of Computing's 303: Systems Verification.

\paragraph{} This reference also contain several contributions\footnote{List of contributors in no particular order.} by Michael Akintunde. Thank you!

\paragraph{} The ordering of this reference may not correspond to the sequence introduced in the lectures, lecture slides and notes. This order is how I feel I would understand the topic better.

\begin{footnotesize}
\paragraph{License} This reference is made publicly available under the MIT License. You should not have paid anyone money in exchange for this document. If you have paid someone for it, well too bad. The source code for this document can be found public available on my Github repository\footnote{\href{https://github.com/mauris/written}{https://github.com/mauris/written}}. If you wish to help improve this document, feel free to open an issue on the Github repository.
\end{footnotesize}

\tableofcontents
\newpage

\section{Introduction}

\paragraph{} There are two ways to know whether we're building a piece of software correctly: validation and verification.

\paragraph{Validation} To assure that the product, service, or system meets the requirements of the customer or other stakeholders.
\paragraph{Verification} To evaluate if the product, service or system compiles with regulation, requirement, specification or imposed conditions.

\paragraph{} We want to be able to automatically verify that a software or system. In order to do that define that a system has to satisfy a set of properties:

\begin{enumerate}
\item We would like to check if a system $S$ satisfies a property $P$.
\item We build an "appropriate" model $M_S$ for $S$ that represents all possible computations of interest of $S$.
\item We define an appropriate formula $\phi P$ capturing property $P$.
\item We can then check automatically if $\phi P$ is satisfied on $M_S$: i.e. $$M_S \models \phi P$$
\end{enumerate}

\section{Modal Languages}

\paragraph{} Modal logic offers:
\begin{enumerate}
\item A natural way of handling the concepts of temporal flows, knowledge, belief, necessity, possibility etc.
\item A clear and natural semantics.
\item A heritage of techniques for proving meta-logical results about modal systems\footnote{such as decidability, completeness, computational complexity, etc.}
\end{enumerate}

\subsection{Syntax vs Semantics}

\paragraph{Syntax} has to do with how we write formulas. It defines the logical language that we use when writing formulas.

\paragraph{Semantics} focuses on giving an interpretation to formulas constructed according to the syntax.

\subsection{Syntax}

\paragraph{Modality} We use an extension of propositional calculus by an operator, $\Box$, that we refer to as \textit{modality} or \textit{box}.

\begin{defn}\label{defn:Syntax} Assume a set of propositional variables $P$, the set of $\mathcal{L}$ of valid formulas of propositional modal logic defined by: \end{defn}

\begin{itemize}
\item \textbf{true} $\in \mathcal{L}$.
\item any $p \in P \implies p \in \mathcal{L}$.
\item If $\phi, \psi \in \mathcal{L}$, then $\lnot \phi, \phi \land \psi, \Box \phi \in \mathcal{L}$.
\item Nothing else is in $\mathcal{L}$,
\end{itemize}

We refer $\phi, \psi,...$ to arbitrary formulas in $\mathcal{L}$ and $p, q, ...$ to atoms in $P$. It is convenient to use other connectives derived from Definition \ref{defn:Syntax}:

\begin{itemize}
\item \textbf{false} $\equiv$ $\lnot$\textbf{true}.
\item $\phi \lor \psi \equiv \lnot(\lnot\phi \land \lnot\psi)$.
\item $\phi \implies \psi \equiv \lnot\phi\lor\psi \equiv \lnot(\phi \land \lnot\psi)$.
\item $\Diamond\phi \equiv \lnot\Box\lnot\phi$.
\end{itemize}

\noindent The modal operator $\Diamond$ is the dual of $\Box$ and is read as "diamond".

\subsection{Modality Readings}

\paragraph{Intuitive Meaning} We can develop the whole of modal theory without any reference to the intuitive meaning of the modalities. There are some meanings of the modal box such as:

\medskip 
\noindent
{\small
\bgroup
\def\arraystretch{1.5}
\begin{tabular}{ | p{1.8cm} || p{2.4cm} | p{2.4cm} | }
\hline
\bf Meaning of $\Box$ & \bf Reading of $\Box\phi$ & \bf Reading of $\Diamond\phi$ \\
\hline
\hline

Temporal & Forever in the future $\phi$ holds & Sometime in the future $\phi$ will hold \\
\hline

Epistemic & It is known that $\phi$ holds & It is considered (epistemically) possible that $\phi$ will hold \\
\hline

Doxastic & It is believed that $\phi$ holds & It is considered (doxastically) possible that $\phi$ will hold \\
\hline

Deontic & $\phi$ is obligatory & $\phi$ is allowed \\
\hline

Provability & It is provable that $\phi$ & It is consistent that $\phi$ \\
\hline

Necessitation & It is necessary that $\phi$ holds & It is possible that $\phi$ hold \\
\hline

\end{tabular}
\egroup
}

\begin{exmp}In the \textit{Epistemic} meaning, $\Box\Box\phi$ is read as "$\phi$ is known to be known".\end{exmp}
\begin{exmp}In the \textit{Epistemic} meaning, $\Box\Diamond\phi$ is read as "$\phi$ is known to be regarded as possible".\end{exmp}
\begin{exmp}In the \textit{Temporal} meaning, $\phi \implies \Box\phi$ is read as "If $\phi$ is true now, then $\phi$ will always be true."\end{exmp}
\begin{exmp}In the \textit{Temporal} meaning, $\Box\phi \implies \Diamond\phi$ is read as "If $\phi$ is forever true, then $\phi$ will be true sometime in the future." \end{exmp}

\paragraph{Combination of Operators} It is also possible to combine $\Box$ operators with different interpretation. Let $\Box_T$ be the modality with Temporal meaning and $\Box_K$ be the modality with Epistemic meaning. $\Box_T\Box_K\phi$ is read as "It is forever known that $\phi$ holds."

\subsection{Kripe Frame}

\begin{defn}A Kripe frame $F$ is a pair $F = (W, R)$ where $W \neq \emptyset$ is a set of possible worlds and $R \subset W \times W$ is a relation $R$ defined on $W$.\end{defn}

\paragraph{} We indicate elements of W as $w$, $w'$, etc. or $s$, $s'$, etc. when in the context of temporal logic These are also called states or points.

\begin{exmp}$(N, suc)$, the set of natural numbers with the relation successor, is a Kripke frame.\end{exmp}

\paragraph{} Since Kripke frames are unvalued structures (i.e. we cannot evaluate formulas on them), we need Kripke models to evaluate formulas.

\subsection{Kripe Models}

\begin{defn}A Kripke model $M$ is a pair $M = (F, \pi)$ where $F$ is a Kripe frame and $\pi: P \mapsto \mathcal{P}(W)$ is a valuation for the atoms. \end{defn}

\paragraph{} Intuitively, $\pi(p)={w_1, w_2}$ represents the fact that the atom $p$ is true at states $w_1$ and $w_2$, and is false at $W \backslash \{w_1, w_2\}$. 

\paragraph{} By slight abuse of notation, we can indicate a Kripke model $M$ as a triple $M = (W, R, \pi)$, where $F = (W, R)$ is its underlying Kripke frame. 

\subsection{Satisfaction}

\begin{defn}The satisfaction of a formula $\phi\in\mathcal{L}$ at a world $w\in W$ of a model $M$ (formally $(M, w) \models \phi$) is inductively defined as follows:\end{defn}

\begin{itemize}
\item $(M, w) \models $ \textbf{true}
\item $(M, w) \models p \impliedby w \in \pi(p)$
\item $(M, w) \models \lnot\phi \impliedby \lnot((M, w)\models \phi)$
\item $(M, w) \models \phi \land \psi \impliedby ((M,w) \models \phi) \land ((M, w) \models \psi)$
\item $(M, w) \models \Box\psi \impliedby\\ (\forall w' \in W (w R w' \implies (M, w') \models \psi)$
\item $(M, w) \models \Diamond\psi \impliedby\\ (\exists w' \in W (w R w' \land (M, w') \models \psi)$
\end{itemize}

\noindent We read $(M, w) \models \phi$ as "$\phi$ is true at $w$ in model $M$".

\section{Linear Temporal Logic}

\paragraph{} LTL asssumes time is a linear sequence of determined discrete events. 

\begin{itemize}
\item The modal box $\Box$ is written as $G$ representing "forever in the future (globally)"
\item Its dual $\Diamond$ is represented by $F$ representing "at some point in the future."
\end{itemize}

\subsection{Operators}

\begin{itemize}
\item $G\phi$ represent situations in which "$\phi$ is forever \textbf{true} from now on."
\item $F\phi$ encode situations in which "$\phi$ will become \textbf{true} at some point in future\footnote{includes the current point in time in the context of this literature. See Section \ref{subsubsec:LTLSemanticsPath} for formal definition.}." Since $F$ is the dual of $G$, $F\phi ::= \lnot G\lnot\phi$.
\item $X\phi$ represent situations where "$\phi$ holds at the next time instant."
\item $U$ is a binary operator, written in the form $\phi U \psi$ which represents that "$\phi$ holds until $\psi$ becomes \textbf{true} (at least once)." Note that even after $\psi$ becomes \textbf{true}, $\phi$ doesn't necessary have to become false (i.e. it can continue to hold).
\item $R$ is also a binary operator, written in the form $\phi R \psi$, representing "$\psi$ holds until $\phi$ becomes \textbf{true} (or $\phi$ releases $\psi$)." Since $R$ is the dual of $U$, $\phi R \psi ::= \lnot(\lnot\psi U\lnot\phi)$
\item $W$ is a weaker version of $U$ (weak until) which relaxes the constraint that $\psi$ needs to be true at some point in time, defined as $\phi W\psi ::= (\phi U \psi)\lor G\phi$.
\end{itemize}

\subsection{Syntax}

\begin{defn}The syntax of LTL is given by the following Backus–Naur form (BNF):\end{defn}

\begin{align*}
\phi ::= p \\
\lnot\phi \\
\phi \land \phi  \\
X\phi \\
G\phi \\
\phi U \phi
\end{align*}


\subsection{Semantics}

\begin{defn}A model for LTL is a Kripke model $M = (W, R, \pi)$ st $R$ is a serial relation. A path $\rho$ in a model is a \textit{infinite sequence of states} $s_0, s_1, ...$ st $\forall i \geq 0,\ (s_i, s_{i+1})\in R$. \end{defn}

A path $\rho$ represents a possible evolution of the system. A state may belong to more than one path, i.e. for any state there may be more than one successor depending on which path we are considering.

\begin{defn}$\rho^i$ indicates the suffix of path $\rho = s_0,s_1,...$, Since a path $\rho$ is infinite, its suffix $\rho^i$ is also infinite and hence is also a path.\end{defn}

\subsubsection{Satisfaction on Paths}\label{subsubsec:LTLSemanticsPath}
\begin{defn}\label{defn:ltlsatisfaction}
Given a formula $\phi$, a model $M$ and a path $\rho = s_0, s_1, s_2,...$ on $M$, satisfaction for the LTL connectives is defined as follows:
\end{defn}

\begin{itemize}
\item $(M, \rho) \models p \Longleftrightarrow s_0 \in \pi(p)$
\item $(M, \rho) \models \lnot\phi \Longleftrightarrow \lnot ((M, \rho)\ \models \phi)\\ \Longleftrightarrow (M, \rho) \not\models \phi$
\item $(M, \rho) \models \phi\land\psi \Longleftrightarrow (M, \rho) \models \phi \land (M, \rho) \models \psi$
\item $(M, \rho) \models X\phi \Longleftrightarrow (M, \rho^1) \models \phi$
\item $(M, \rho) \models G\phi \Longleftrightarrow \forall i \geq 0\ (M, \rho^i) \models \phi$
\item $(M, \rho) \models \phi U\psi \Longleftrightarrow\\ \exists j \geq 0\ \lbrack\forall k \in [0, j)\ (M, \rho^k) \models \phi \land (M, \rho^j) \models \psi\rbrack$
\end{itemize}

\paragraph{} Using Definition \ref{defn:ltlsatisfaction}, we can derive the following additional satisfactions:

\begin{itemize}
\item $(M, \rho) \models F\phi \Longleftrightarrow \exists i \geq 0\ (M, \rho^i) \models \phi$
\item $(M, \rho) \models \psi R\phi \Longleftrightarrow\\ \exists j \geq 0\ \lbrack \forall k \in [0, j]\ (M, \rho^k) \models \phi \land (M, \rho^j) \models \psi\rbrack\\ \lor \forall i \geq 0\ (M, \rho^i) \models \phi$
\end{itemize}

\subsubsection{Satisfaction on States}
\begin{defn}
Given a formula $\phi$, a model $M$ and a state $s$ in $M$, $\phi$ is true at $s$ in $M$ (written $(M, s) \models \phi$, if $\forall$ paths $\rho$ originating from $s$, we have $(M, \rho) \models \phi$. Formally, $$\forall \rho(s): (M, \rho) \models \phi \implies (M, s) \models \phi$$
\end{defn}

\paragraph{} In the case of LTL, the standard modal satisfaction definition on states involve quantification over \textit{all possible futures}.

\subsection{Equivalences}

\paragraph{} There are several useful equivalences that hold in LTL:

\begin{itemize}
\item $\lnot X\phi \equiv X \lnot\phi$
\item $G(\phi \land \psi) \equiv G\phi \land G\psi$
\item $F(\phi \lor \psi) \equiv F\phi \lor F\psi$
\item $G\phi \equiv \text{false}\ R \phi\\ \equiv \lnot (\text{true}\ U \lnot\phi)$
\item $F\phi \equiv \text{true}\ U \phi\\ \equiv \lnot (\text{false}\ R \lnot\phi)$
\item $\phi U \psi \equiv \phi W \psi \land F\psi$
\item $\phi W \psi \equiv (\phi U \psi) \lor G\phi\\
 \equiv \phi U (\psi \lor G\phi)\\
 \equiv \psi R (\psi \lor \phi)$
\item $\psi R \phi \equiv \phi W (\phi \land \psi)\\
 \equiv (\phi U (\phi \land \psi)) \lor G(\phi \land \psi)$
\end{itemize}

\subsection{Non-Equivalences}

\paragraph{} Different forumlas may provide different semantic meanings. For example, we see that $Fp \implies Fq \not\equiv F(p \implies q)$:

\begin{itemize}
\item[] We assume that $Fp \implies Fq \equiv F(p \implies q)$. We use the definition of $\implies$ to derive:\\ $Fp \implies Fq \equiv \lnot Fp \lor Fq$ and\\ $F(p \implies q) \equiv F(\lnot p \lor q) \equiv F\lnot p \lor Fq$.
\item[] To reach a contradiction, we need a model that satisfies one formula but not the other. Since we need $Fq$ to be false (otherwise $Fq$ would make both formulae true), we must never reach a point in the model where $q$ holds. We choose to eliminate $q$ from any counterexample we construct.
\item[] We're left with $\lnot Fp$ and $F\lnot q$. One must be true and the other false.
\item[] We can try to make $\lnot Fp$ true and $F\lnot p$ false, but in order to make $F\lnot p$ false, we need to $Gp$ in the model to be true, which is impossible when we're already making $\lnot Fp$ true.
\item[] Alternatively, we make $\lnot Fp$ false and $F\lnot p$ true in the same model. We construct a model such that $p$ starts off as true, and at some other point $p$ becomes false. 
\end{itemize}

\noindent With the following model $\mathcal{M}$: for all models $Fp \implies Fq \not\equiv F(p \implies q)$.

\begin{figure}[H]
\centering
\includegraphics[width=0.2\textwidth]{graphics/ltl-non-equiv-example1.png}
\caption{The constructed model $\mathcal{M}$ st given the sequence $\rho = w_1 w_2^+$ we have $(\mathcal{M}, \rho) \not\models Fp \implies Fq$, but $(\mathcal{M}, \rho) \models F(p \implies q)$.}
\end{figure}

\paragraph{} By the counterexample, we proved that there exists a model $\mathcal{M}$ and a path $\rho$ st $(\mathcal{M}, \rho) \not\models (Fp \implies Fq) \land (\mathcal{M}, \rho) \models F(p \implies q)$. Hence $Fp \implies Fq \not\equiv F(p \implies q)$.

\section{Computation Tree Logic}

\paragraph{} LTL allows us to talk about the temporal evolution of a system, but sometimes we would like to check whether or not something happens in one path but not in all. CTL accommodates this need. CTL's syntax allows one to \textit{quantify explicitly} over paths.

\subsection{Syntax}
\begin{defn}The syntax of CTL is given by the following BNF:\end{defn}

\begin{align*}
\phi ::= p \\
\lnot\phi \\
\phi \land \phi  \\
EX\phi \\
EG\phi \\
E(\phi U \phi)
\end{align*}

\subsection{Operators}\label{subsec:Operators}

\paragraph{} In CTL we have two different path quantification modifiers: $E$ and $A$. $E$ encodes an existential quantification on paths while $A$ encodes a universal quantification on paths. 

\begin{itemize}
\item $EX\phi$: "there exists a path from the current state st at the next state $\phi$ holds."
\item $EG\phi$: "there exists a path from the current state st $\phi$ holds forever in the future."
\item $E(\phi U\psi)$: "there exists a path from the current state st $\phi$ holds until $\psi$ becomes true."
\end{itemize}

\paragraph{} In CTL, dual operators can also be defined: 
\begin{itemize}
\item $AX\phi ::= \lnot EX \lnot\phi$
\item $EX\phi ::= \lnot AX \lnot\phi$
\item $AG\phi ::= \lnot EF \lnot\phi$
\item $AF\phi ::= \lnot EG \lnot\phi$
\item $EF\phi ::= \lnot AG \lnot\phi$
\item $EG\phi ::= \lnot AF \lnot\phi$
\end{itemize}

\subsection{Semantics}
\subsubsection{Satisfaction}
\begin{defn}
Given a formula $\phi$, a model $M = (W, R, \pi)$ and a state $s$ in $M$, the satisfaction of $\phi$ at $s$ in $M$ (written $(M, s) \models \phi$) is defined inductively as follows:
\end{defn}

\begin{itemize}
\item $(M, s) \models p \Leftrightarrow s \in \pi(p)$
\item $(M, s) \models EX\phi \Leftrightarrow\\ \exists \rho = s_0, s_1, ... \land s = s_0: (M, s_1) \models \phi$
\item $(M, s) \models EG\phi \Leftrightarrow\\ \exists \rho = s_0, s_1, ... \land s = s_0: \forall i \geq 0: (M, s_i) \models \phi$
\item $(M, s) \models EF\phi \Leftrightarrow\\ \exists \rho = s_0, s_1, ... \land s = s_0: \exists i \geq 0: (M, s_i) \models \phi$
\item $(M, s) \models E(\phi U\psi) \Leftrightarrow\\ \exists \rho = s_0, s_1, ... \land s = s_0:\\ \exists j \geq 0\  \lbrack\forall k \in [0, j): (M, s_k) \models \phi \land (M, s_j) \models \psi\rbrack$
\item $(M, s) \models E(\psi R\phi) \Leftrightarrow\\ \exists \rho = s_0, s_1, ... \land s = s_0:\\ \exists j \geq 0\  \lbrack\forall k \in [0, j]: (M, s_k) \models \phi \land (M, s_j) \models \psi\rbrack\\ \lor \forall i \geq 0\ (M, s_i) \models \phi$
\item $(M, s) \models AG\phi \Leftrightarrow\\ \forall \rho = s_0, s_1, ... \land s = s_0: \forall i \geq 0: (M, s_i) \models \phi$
\item $(M, s) \models AF\phi \Leftrightarrow\\ \forall \rho = s_0, s_1, ... \land s = s_0: \exists i \geq 0: (M, s_i) \models \phi$
\item $(M, s) \models A(\phi U\psi) \Leftrightarrow\\ \forall \rho = s_0, s_1, ... \land s = s_0:\\ \exists j \geq 0\  \lbrack\forall k \in [0, j): (M, s_k) \models \phi \land (M, s_j) \models \psi\rbrack$
\item $(M, s) \models A(\psi R\phi) \Leftrightarrow\\ \forall \rho = s_0, s_1, ... \land s = s_0:\\ \exists j \geq 0\  \lbrack\forall k \in [0, j]: (M, s_k) \models \phi \land (M, s_j) \models \psi\rbrack\\ \lor \forall i \geq 0\ (M, s_i) \models \phi$
\end{itemize}

\subsubsection{Skolemization}

\paragraph{} $AFEG\phi \neq EGAF\phi$. The $EG\phi$ in $AFEG\phi$ is dependent on the path taken by the quantifier $AF$ and hence cannot be interchanged without changing its semantic meaning.

\subsection{Semantic Equivalence}

\paragraph{} The following are some semantic equivalence apart those seen in Section \ref{subsec:Operators}:

\begin{itemize}
\item $AG\phi \equiv \phi \land AXAG\phi$
\item $EG\phi \equiv \phi \land EXEG\phi$
\item $AF\phi \equiv \phi \lor AXAF\phi$
\item $EF\phi \equiv \phi \lor EXEF\phi$
\item $A(\phi U \psi) \equiv \psi \lor (\phi \land AX[A(\phi U \psi)])$
\item $E(\phi U \psi) \equiv \psi \lor (\phi \land EX[E(\phi U \psi)])$
\item $EG\phi \equiv \lnot A(\text{true}\ U \lnot\phi)$
\item $EF\phi \equiv E(\text{true}\ U \phi)$
\item $AG\phi \equiv \lnot EF \lnot\phi \equiv \lnot E(\text{true}\ U \lnot\phi)$
\item $AF\phi \equiv \lnot EG \lnot\phi \equiv A(\text{true}\ U \phi)$
\item $A(\phi U \psi) \equiv \lnot E(\lnot\psi U \lnot(\phi\lor\psi)) \land \lnot EG(\lnot \psi)\\
 \equiv \lnot E(\lnot\psi U \lnot(\phi\lor\psi)) \land AF\psi\\
 \equiv \lnot A(\phi R \psi)$
\end{itemize}

\subsection{Non-Equivalences}

\paragraph{} Formulas can differ in what they specify. For example, $AGAF\phi \not\equiv AFAG\phi$.

\begin{itemize}
\item $AGAF\phi$ states that however far I go into any path, I will always find another $\phi$ in all the paths from there. 
\item $AFAG\phi$ states that in all the paths, eventually you only find paths where $\phi$ always holds.
\end{itemize}


\section{Model Checking}

\paragraph{} Suppose we have a system $S$ and a property $P$ representing the specification of the system. Assume that we can build a Kripke model $M_S = (W, R, \pi)$ representing all possible computations of $S$. 

\begin{itemize}
\item $W$ contains all the possible computational states of $S$.
\item $R$ is the temporal relation between states that represents all temporal transitions in the system.
\end{itemize}

\paragraph{} Assume that property $P$ can be represented as a logical formula $\phi_P$. If $M_S$ faithfully encodes $S$ and $\phi_P$ captures $P$, then checking whether $S$ satisfies $P$ can be solved by checking if

$$M_S \models \phi_P$$

\paragraph{} However, there may still be some questions we need to address:

\begin{itemize}
\item Are the temporal logics (LTL / CTL) expressive enough to express the kind of specifications $\phi_P$ we are likely to encounter?
\item How do we build $M_S$?
\item Won't $M_S$ have too many states (possibly infinite) for us to reason able?
\end{itemize}

\subsection{Specifications in LTL and CTL}

\paragraph{} There are some statements that can be naturally expressed in LTL, CTL or both. Other statements may never be able to express in one of the temporal logics.

\begin{itemize}
\item "A faulty state is never reached":\\ LTL: $\lnot F\ faulty$\\ CTL: $\lnot EF\ faulty$
\item "It is impossible to get to a state where $started$ holds but $ready$ does not":\\ LTL: $G\lnot(started \land \lnot ready)$\\ CTL: $AG\lnot(started \land \lnot ready)$
\item "It is possible (i.e. there is a possible computation in which this happens) to get to a state where $started$ holds but $ready$ does not.":\\ LTL: we cannot express this!\\CTL: $EF(started, \lnot ready)$
\item "Every request will get acknowledged":\\ LTL: $G(req \implies F\ ack)$\\ CTL: $AG(req \implies AF\ ack)$
\item "An event $p$ happens infinitely often":\\ LTL: $GF\ p$\\ CTL: $AGAF\ p$
\end{itemize}

\subsection{Case Study: Mutual Exclusion}

\subsubsection{Introduction}

\paragraph{} We consider a key implementation of distributed systems - mutual exclusion. There are three main property of a mutual exclusion system:

\begin{itemize}\label{list:MutexProperties}
\item \textbf{Safety}: Only one process may have access to the same resource at any time.
\item \textbf{Liveness}: Whenever a process requests for the access to the resource, it will \textit{eventually} be granted the access.
\item \textbf{Non-blocking}: A process can always make a request to the use the resource.
\end{itemize}

\paragraph{} How do we build a system that satisfies all three properties? i.e. what is the simplest example of a system that satisfies the properties?

\paragraph{Naive Implementation} One method is to let the processes take turn (with timeout) i.e. time slicing. This is inefficient and would not scale if we need to add more processes. 

\paragraph{} We add another specification - \textbf{No strict sequencing}: No ordering exists in resource access.

\paragraph{} To model these properties of the system in temporal logic, we assign some propositions:

\begin{itemize}
\item Process $i$ is in critical section ($c_i$) when it is using the resource.
\item Process $i$ is in the trying state ($t_i$) when it is making the request to use the resource.
\item Process $i$ is in outside of critical section ($n_i$) when it is not using the resource nor requesting for access to it.
\end{itemize}

\paragraph{} In this example, we assume that there are two processes in the world and each process can only be in one of the three states written above.

\subsubsection{LTL Specification}

\paragraph{} We attempt to model in LTL:

\begin{itemize}
\item \textbf{Safety}: $G\lnot(c_1 \land c_2)$
\item \textbf{Liveness}: $G(t_1 \implies Fc_1) \land G(t_2 \implies Fc_2)$
\item \textbf{Non-blocking}: We cannot express this in LTL.
\item \textbf{No strict sequencing}: We cannot express this directly, but we can express a formula that implies the negation of no strict sequencing: $G(c_1 \implies (c_1\ W\ (\lnot c_1 \land \lnot c_1\ W\ c_2)))$.\footnote{It says that everytime we get to a $c_1$ state, either we remain in $c_1$ forever or we need to go through a $c_2$ state before getting to $c_1$ again.}
\end{itemize}

\subsubsection{CTL Specification}

\paragraph{} We attempt to model in CTL:

\begin{itemize}
\item \textbf{Safety}: $AG\lnot(c_1 \land c_2)$
\item \textbf{Liveness}:\\ $AG(t_1 \implies AFc_1) \land AG(t_2 \implies AFc_2)$
\item \textbf{Non-blocking}: 
\item \textbf{No strict sequencing}: 
\end{itemize}

\subsection{Model Building}

\paragraph{} Now that we have a set of formal specifications, we can try to build its model and check whether the model satisfies the specifications given above. 

\begin{figure}[H]
\centering
\includegraphics[scale=0.36]{graphics/mutex-model1.png}
\caption{A model of the mutual exclusion system for process 1 and 2. This model is a possible implementation for a system that satisfies mutual exclusion. The initial state is $s_0$.}\label{fig:MutexModel1}
\end{figure}

\paragraph{} We now check if our \textit{LTL specifications} are satisfied at the initial state $s_0$, i.e. $(M, s_0) \models \phi$, where $\phi$ expresses our four specifications in Section \ref{list:MutexProperties}. We find that:

\begin{itemize}
\item Safety is satisfied: There is no state in the model such that $c_1, c_2$ hold in the same state.
\item Liveness is not satisfied: check the path $s_0s_1s_3s_7s_1...$
\item Non-blocking is not expressible in LTL.
\item The negation of no-strict sequencing is not satisfied, hence the property of no-strict sequencing is satisfied: check the path $s_0s_5s_3s_4s_5s_3s_4$
\end{itemize}

\paragraph{} There are also a few assumptions we are making in this diagram:

\begin{enumerate}
\item Is the amount of time spent in each state unit time or variable time? If it is fixed time, how can we allow process 1 to stay in the critical section as long as it needs while process 2 doesn't try? i.e. do we need to add a loop from $s_2$ to itself and a loop from $s_6$ to itself?
\item A process $i$ is either in $n_i$, $t_i$ or $c_i$ state. We take the assumption that two or more states for a single process cannot happen in the same state of the model.
\end{enumerate}

\paragraph{} The way we represented non-determinism may result in one process always getting preference over the other: i.e. at $s_3$, how do we choose which state to advance to: $s_4$ or $s_7$? We need to revise the model.

\begin{figure}[H]
\centering
\includegraphics[scale=0.36]{graphics/mutex-model2.png}
\caption{The revised model (from Fig. \ref{fig:MutexModel1}) of the mutual exclusion system for process 1 and 2. We split $s_3$ into two so that we can fix the non-determinism and hence satisfying liveness.}\label{fig:MutexModel2}
\end{figure}

\paragraph{} We find that the model in Figure \ref{fig:MutexModel2} satisfies the three properties:

\begin{itemize}
\item Safety is remains satisfied.
\item Liveness is now satisfied.
\item The negation of no-strict sequencing is not satisfied, hence the property of no-strict sequencing remains satisfied.
\end{itemize}

\subsection{Fairness}

\paragraph{} In Figure \ref{fig:MutexModel1}, we see that we had paths in which some requests (states with $t_i$) that would never be served (reaching a state with $c_i$) by some "unfair" executions. To avoid this situation, we need to forbid these paths to happen in the model.

\paragraph{} As shown in Figure \ref{fig:MutexModel2}, we split $s_3$ into $s_3$ and $s_8$ in the model to hardwire order of requests by using appropriate transitions. In some sense we can think of considering the previous path in addition to the current state to determine the appropriate transitions. 

\paragraph{} To solve such a problem, we introduce what we call "fairness constraints". These limits the paths considered to the ones in which some formulas are true indefinitely often. The argument of a fairness constraint can be any formula.

\paragraph{} In the Mutual Exclusion example, we could introduce the fairness constraint $(c_1 \lor c_2)$ representing the fact that along any path, we need to serve either process infinitely often. This can be represented by adding fairness constraint statements in automatic model checkers such as NuSMV.

\paragraph{} There are two classes of fairness constraints that are usually considered:

\begin{itemize}
\item Justice Constraints: "Formula $\phi$ is true infinitely often over any path."
\item Compassion Constraints: "If formula $\phi$ is true infinitely often over any path, then $\psi$ is also true infinitely often."
\end{itemize}

\paragraph{} Note that compassion constraints impose the formula $GF\phi \implies GF\phi$ in all states considered (as it needs to be satisfied in all paths considered). We can express both justice and compassion constraints in NuSMV.

\section{Explicit Model Checking}

\paragraph{} We need an algorithmic way of computing $(M, s_0) \models \phi$ and we need it to be efficient. Turns out it is convenient to compute the whole set of states $[\phi]_M$ on which $\phi$ is true. By definition,

$$s_0 \in [\phi]_M \equiv (M, s_0) \models \phi$$

\subsection{Labelling Algorithm}

\paragraph{} We label the states where the subformulas of $\phi$ are satisfied starting from the smallest subformulas and working outwards towards $\phi$. 

\begin{itemize}
\item Input: A model $M$ and a CTL formula $\phi$.
\item Output: The set of states in $M$ which satisfies $\phi$ (i.e. $[\phi]_M$).
\end{itemize}

\subsubsection{Labelling Propositional Variables}

\paragraph{} First, we consider all propositional variables in $\phi$ and label the states in which they hold. 

\paragraph{} For example, if $\phi = a \land b$, we label all the states where $a$ hold and also all the states where $b$ hold.

\subsubsection{Labelling Subformulas}

\paragraph{} Next, $\forall s$ state and $\forall\psi$ subformulas of $\phi$ (including $\phi$), we perform the following labelling:

\begin{itemize}
\item $\lnot\psi$: label state $s$ with $\lnot\psi$ where the state $s$ is not labelled with $\psi$.
\item $\psi_1 \land \psi_2$: label state $s$ with $\psi_1 \land \psi_2$ if state $s$ has been labelled with both $\psi_1$ and $\psi_2$.
\end{itemize}

\paragraph{} With the two definitions, we can also easily label any OR operators using double negation and De Morgan's laws.

\paragraph{} In the remaining parts of the labelling algorithm, we also consider the minimal implementation by taking advantage of equivalences and duality. Hence we will only discuss the implementation of $EX\psi$, $AF\psi$ and $E(\psi_1\ U\ \psi_2)$ as an adequate set of connectives for CTL. All others can be written in terms of these connectives.

\subsubsection{Labelling $EX\psi$}

\paragraph{} We label any state with $EX\psi$ if \textit{one of its successors} is labelled with $\psi$.

\subsubsection{Labelling $AF\psi$}

\paragraph{} If any state $s$ is labelled with $\psi$, we label it with $AF\psi$. Then we repeatedly label any state with $AF\psi$ if \textit{all of its successors} are labelled with $AF\psi$ until there is no more new labelling.

\subsubsection{Labelling $E(\psi_1\ U\ \psi_2)$}

\paragraph{} If any state $s$ is labelled with $\psi_2$, label it with $E(\psi_1\ U\ \psi_2)$. Then we repeatedly label any state with $E(\psi_1\ U\ \psi_2)$ if it is labelled with $\psi_1$ and \textit{at least one of it successor} is labelled with $E(\psi_1\ U\ \psi_2)$ until there is no more new labelling.

\subsubsection{Algorithm Output}

\paragraph{} When all the labellings are complete, we output the states which are labelled $\phi$ as the answer. 

\subsubsection{Variants: $EG\psi$}

\paragraph{} Instead of adding labels, we could also start from adding labels to all then removing them. For example, we could label states by using the following algorithm for $EG\psi$:

\begin{enumerate}
\item Label \textit{all} states with $EG\psi$.
\item If any state $s$ is not labelled with $\psi$, we delete the label $EG\psi$ from $s$. 
\item Repeatedly delete label $EG\psi$ from states that has no successors labelled $EG\psi$ until there is no more changes.
\end{enumerate}

\subsection{General Algorithm}

\paragraph{} We attempt to build a more formal and general algorithm that takes a CTL formula $\phi$ and a model $M$ as input and return the set of states $[\phi]_M$ as output. The algorithm will rely on basic set operations and three additional functions to calculate set of states that satisfy subformulas with EX, EU and AF.

\paragraph{} Again, we choose the functions to be implemented for EX, EU and AF because they form an adequate set of connectives for CTL. Alternative implementations are possible. We will see that our function will perform the appropriate rewriting for the other CTL connectives.

\subsubsection{Implementation of $SAT(\phi, M)$}

\begin{lstlisting}[mathescape=true]
function SAT($\phi$, M)
switch $\phi$:
  case $\top$: return S(M) 
    /* S(M) -> all states in model M */
  case $\perp$: return $\emptyset$
  case is atomic: return $\{s \in S(M) | \phi \in L(s)\}$
  case $\lnot\psi$: return S \ SAT($\psi$, M)
  case $\psi_1 \land \psi_2$: return SAT($\psi_1$,M) $\cap$ SAT($\psi_2$,M)
  case $\psi_1 \lor \psi_2$: return SAT($\psi_1$,M) $\cup$ SAT($\psi_2$,M)
  case $\psi_1 \rightarrow \psi_2$: return SAT($\lnot\psi_1\lor\psi_2$, M)
  case $AX\psi$: return SAT($\lnot EX\lnot\psi$, M)
  case $A(\psi_1\ U\ \psi_2)$: return
    SAT($\lnot(E[\lnot\psi_2\ U\ (\lnot\psi_1\land\lnot\psi_2)]\lor EG\lnot\psi_2)$, M)
  case $EF\psi$: return SAT($E(\top\ U\ \psi)$, M)
  case $EG\psi$: return SAT($\lnot AF\lnot\psi$, M)
  case $AG\psi$: return SAT($\lnot EF\lnot\psi$, M)
  case $EX\psi$: return SATex($\psi$, M)
  case $AF\psi$: return SATaf($\psi$, M)
  case $E(\psi_1\ U\ \psi_2)$: return SATeu($\psi_1$, $\psi_2$, M)
end switch
end function
\end{lstlisting}

\paragraph{} In the implementation of $SAT(\phi, M)$ above, we can see that rewritings are done using recursive calls to itself with a rewritten version of the formula.

\subsubsection{Auxiliary Function $SATex(\phi, M)$}

\begin{lstlisting}[mathescape=true]
function SATex($\phi$, M)
  local var X, Y
  X := SAT($\phi$, M)
  Y := $\{s_0 \in S(M)\ |\ \exists s_1 \in X: s_0 \rightarrow s_1\}$
  return Y
end function
\end{lstlisting}

\paragraph{} Notice the statement $X := SAT(\phi, M)$ handles complex formula in $\phi$ recursively. 

\subsubsection{Auxiliary Function $SATaf(\phi, M)$}

\begin{lstlisting}[mathescape=true]
function SATaf($\phi$, M)
  local var X, Y
  X := S(M)
  Y := SAT($\phi$, M)
  repeat until X = Y:
    X := Y
    Y := $Y \cup \{s\ |\ \forall s'\text{ successors of } s: s' \in Y\}$
  end repeat
  return Y
end function
\end{lstlisting}

\subsubsection{Auxiliary Function $SATeu(\phi, \psi, M)$}

\begin{lstlisting}[mathescape=true]
function SATeu($\phi$, $\psi$, M)
  local var W, X, Y
  W := SAT($\phi$, M)
  X := S(M)
  Y := SAT($\psi$, M)
  repeat until X = Y:
    X := Y
    Y := $Y \cup (W \cap \{s\ |\ \exists s'\text{ successor of }s: s' \in Y\})$
  end repeat
  return Y
end function
\end{lstlisting}


\end{multicols}
\end{document}